\documentclass[a4paper,10pt,twoside]{article}

\usepackage[english]{babel}
\usepackage{amssymb}
\usepackage{fancyhdr}
\usepackage{graphicx}
\usepackage[ocgcolorlinks]{hyperref}
\usepackage{todonotes}
\usepackage{caption}
\usepackage{subcaption}

\pagestyle{fancy}
\fancyhead{}
\fancyfoot{}
\fancyhead[RE,LO]{de Kok \& Methenitis}
\fancyhead[RO,LE]{Week 3}
\fancyfoot[RE]{\thepage$\quad \square$}
\fancyfoot[LO]{$\square \quad$\thepage}

\title{Report for Week 3 \\\normalsize $k$-means \& Bag-of-Visual words\\ or how to create a Google Goggles-like system}

\author{Patrick de Kok (5640318) \and Georgios Methenitis (10407537)}

\begin{document}
\maketitle
\thispagestyle{empty}

\section{Individual steps of $k$-means}

\section{Running your implemented $k$-means}

\section{Color-based image segmentation}

\section{$k$-means, SIFT \& Bag-of-Words: calculating storage for raw SIFT}
Each SIFT vector has 128 elements.  Each element is represented by an integer\footnote{The binary invoked to extract the SIFT vectors returns integers.  However, the \texttt{read\_sift\_from\_file(sift\_path)} function interprets these as Python \texttt{float} objects.  In the standard Python implementation, CPython, a \texttt{float} always consists of 64 bits, which is 8 bytes.  This would result in quadrupling all final answers.}.  Assuming each integer consists of 2 bytes and there is no storage overhead for the vector, 1 SIFT vector needs 256 bytes of storage.  To store all 50,000 vectors that belong to one image, 12,800,000 bytes = $12.8 \cdot 10^6 B = 12.8 \mbox{ MB} \approx 12.21 \mbox{ MiB}$ is needed.  For Facebook's daily addition of new images, one needs $12.8 \cdot 10^6 \cdot 350 \cdot 10^6 \mbox{ B} = 4.48 \cdot 10^{15} \mbox{ B} = 4.48 \mbox{ PB} = 4,480,000,000 \mbox{ MB} = 4272460937.5 \mbox{ MiB} \approx 3.98 \mbox{ PiB}$ of storage.

\section{Visualizing the words on the image, the contents of the words and the Bag-of-Words histograms}
Visual words of five randomly selected images from the dataset have been plotted in Figure~\ref{fig:wordsonimage}.  Note that words tend to be focused on structures that reoccur often, such as lines.  This can be seen in Figure~\ref{sfig:cc85} and~\ref{sfig:as1} at the roof, and in Figure~\ref{sfig:as15} and~\ref{sfig:rc437} at the 3D structures on the wall.

\begin{figure}
  \begin{subfigure}{0.49\textwidth}
    \centering
    \includegraphics[width=\textwidth,height=.3\textheight,keepaspectratio]{randomimage1}
    \caption{\texttt{christ\_church\_000085.png}}
    \label{sfig:cc85}
  \end{subfigure}
  \begin{subfigure}{0.49\textwidth}
    \centering
    \includegraphics[width=\textwidth,height=.3\textheight,keepaspectratio]{randomimage2}
    \caption{\texttt{all\_souls\_000001png}}
    \label{sfig:as1}
  \end{subfigure}
  \begin{subfigure}{0.49\textwidth}
    \centering
    \includegraphics[width=\textwidth,height=.3\textheight,keepaspectratio]{randomimage3}
    \caption{\texttt{all\_souls\_000015png}}
    \label{sfig:as15}
  \end{subfigure}
  \begin{subfigure}{0.49\textwidth}
    \centering
    \includegraphics[width=\textwidth,height=.3\textheight,keepaspectratio]{randomimage4}
    \caption{\texttt{christ\_church\_000275png}}
    \label{sfig:cc275}
  \end{subfigure}
  \begin{subfigure}{0.49\textwidth}
    \centering
    \includegraphics[width=\textwidth,height=.3\textheight,keepaspectratio]{randomimage5}
    \caption{\texttt{radcliffe\_camera\_000437png}}
    \label{sfig:rc437}
  \end{subfigure}
  \begin{subfigure}{0.49\textwidth}
    \centering
    \includegraphics[width=\textwidth,height=.3\textheight,keepaspectratio]{colorbar}
    \caption{Legend.}
    \label{sfig:colorbar}
  \end{subfigure}
  \caption{SIFT vectors that have been assigned to words, plotted on images.  Figure~\ref{sfig:colorbar} functions as a legend for each feature: dots of the same color represent the same SIFT vector.  Its number can be looked up on this bar.}
  \label{fig:wordsonimage}
\end{figure}

The patches of five randomly selected words are shown in Figure~\ref{fig:words}.  The patches do look like each other, when you ignore the differences in hue (but not in intensity), rotation and scale.

\begin{figure}
  \begin{subfigure}{0.49\textwidth}
    \centering
    \includegraphics[width=\textwidth,height=.3\textheight,keepaspectratio]{word1}
    \caption{Word 69}
  \end{subfigure}
  \begin{subfigure}{0.49\textwidth}
    \centering
    \includegraphics[width=\textwidth,height=.3\textheight,keepaspectratio]{word2}
    \caption{Word 65}
  \end{subfigure}
  \begin{subfigure}{0.49\textwidth}
    \centering
    \includegraphics[width=\textwidth,height=.3\textheight,keepaspectratio]{word3}
    \caption{Word 10}
  \end{subfigure}
  \begin{subfigure}{0.49\textwidth}
    \centering
    \includegraphics[width=\textwidth,height=.3\textheight,keepaspectratio]{word4}
    \caption{Word 2}
  \end{subfigure}
  \begin{subfigure}{0.49\textwidth}
    \centering
    \includegraphics[width=\textwidth,height=.3\textheight,keepaspectratio]{word5}
    \caption{Word 38}
  \end{subfigure}
  \caption{Patches for five randomly selected words, as they occur in the images of Figure~\ref{fig:wordsonimage}.}
  \label{fig:words}
\end{figure}

Finally, the bag-of-words-histogram representation of the images of Figure~\ref{fig:wordsonimage} is presented in Figure~\ref{fig:histograms}.

\begin{figure}
  \begin{subfigure}{0.49\textwidth}
    \centering
    \includegraphics[width=\textwidth,height=.3\textheight,keepaspectratio]{histogram1}
    \caption{\texttt{christ\_church\_000085.png}}
  \end{subfigure}
  \begin{subfigure}{0.49\textwidth}
    \centering
    \includegraphics[width=\textwidth,height=.3\textheight,keepaspectratio]{histogram2}
    \caption{\texttt{all\_souls\_000001.png}}
  \end{subfigure}
  \begin{subfigure}{0.49\textwidth}
    \centering
    \includegraphics[width=\textwidth,height=.3\textheight,keepaspectratio]{histogram3}
    \caption{\texttt{all\_souls\_000015.png}}
  \end{subfigure}
  \begin{subfigure}{0.49\textwidth}
    \centering
    \includegraphics[width=\textwidth,height=.3\textheight,keepaspectratio]{histogram4}
    \caption{\texttt{christ\_church\_000275.png}}
  \end{subfigure}
  \begin{subfigure}{0.49\textwidth}
    \centering
    \includegraphics[width=\textwidth,height=.3\textheight,keepaspectratio]{histogram5}
    \caption{\texttt{radcliffe\_camera\_000437.png}}
  \end{subfigure}
  \caption{Frequencies of visual words per image.  Each bar is visualised in the corresponding SIFT vector color.  Figure~\ref{sfig:colorbar} shows these colors.}
  \label{fig:histograms}
\end{figure}

\section{Visualizing Bag-of-Words retrieval}
See Figure~\ref{fig:bow_retrieval} for the ranking.

\begin{figure}
  \begin{subfigure}{0.47\textwidth}
    \centering
    \includegraphics[width=\textwidth]{ranking_h_6}
    \caption{Histogram distance, \texttt{all\_souls\_000006.png}.}
  \end{subfigure}
  \hspace*{\fill}
  \begin{subfigure}{0.47\textwidth}
    \centering
    \includegraphics[width=\textwidth]{ranking_l_6}
    \caption{$\ell_2$ distance, \texttt{all\_souls\_000006.png}.}
  \end{subfigure}
  \begin{subfigure}{0.47\textwidth}
    \centering
    \includegraphics[width=\textwidth]{ranking_h_65}
    \caption{Histogram distance, \texttt{all\_souls\_000065.png}.}
  \end{subfigure}
  \hspace*{\fill}
  \begin{subfigure}{0.47\textwidth}
    \centering
    \includegraphics[width=\textwidth]{ranking_l_65}
    \caption{$\ell_2$ distance, \texttt{all\_souls\_000065.png}.}
  \end{subfigure}
  \begin{subfigure}{0.47\textwidth}
    \centering
    \includegraphics[width=\textwidth]{ranking_h_75}
    \caption{Histogram distance, \texttt{all\_souls\_000075.png}.}
  \end{subfigure}
  \hspace*{\fill}
  \begin{subfigure}{0.47\textwidth}
    \centering
    \includegraphics[width=\textwidth]{ranking_l_75}
    \caption{$\ell_2$ distance, \texttt{all\_souls\_000075.png}.}
  \end{subfigure}
  \caption{Ranking of retrieved images based on different distance measures between histograms of bags of words.}
  \label{fig:bow_retrieval}
\end{figure}

\section{Evaluating Bag-of-Words histograms}
Precision scores of the bag of words retrieval system can be found in Figure~\ref{fig:precision}.  Precision scores of the color histogram retrieval system for the same images can be found in table~\ref{tab:color_precision}.  The precision of the bag of words retrieval system seems to have an optimum for the codebook size.  Smaller codebooks such as 5 and 10 seem to perform at least equally well in most cases as larger codebooks such as 500 and 1000.  No strong conclusions can be made based on the current observations, as only a small number of queries have been examined.  

From the average precision scores, we might be able to conclude that the histogram distance performs better than the $\ell_2$ distance, although there is barely any difference.  

We might also conclude that there are very few images like \texttt{all\_souls\_000075}, as neither measure returns high precision.  

To propose a method to improve the accuracy of the system, we should investigate the current accuracy as well, which has not been done. A method to improve the precision might be to use a different feature descriptor than SIFT.  One might want to look at histograms of oriented gradients, SURF or AGAST.

Comparing the bag of words retrieval system with the color histogram retrieval system, we can observe that only for the precision@5 with heuristic distance measure and precision@10 with the $\ell_2$ distance measure, the color histogram system outperforms the bag of words system with codebook 100 with query image \texttt{all\_souls\_000075}.  In all other cases, the bag of words system outperforms the color histogram system.  This is not that strange; the bag of words system is based on a more advanced feature descriptor method than color histograms.

\begin{figure}
  \begin{subfigure}{0.47\textwidth}
    \centering
    \includegraphics[width=\textwidth]{p5h}
    \caption{Precision@5, histogram distance.}
  \end{subfigure}
  \hspace*{\fill}
  \begin{subfigure}{0.47\textwidth}
    \centering
    \includegraphics[width=\textwidth]{p5l}
    \caption{Precision@5, $\ell_2$ distance.}
  \end{subfigure}
  \begin{subfigure}{0.47\textwidth}
    \centering
    \includegraphics[width=\textwidth]{p10h}
    \caption{Precision@10, histogram distance.}
  \end{subfigure}
  \hspace*{\fill}
  \begin{subfigure}{0.47\textwidth}
    \centering
    \includegraphics[width=\textwidth]{p10l}
    \caption{Precision@10, $\ell_2$ distance.}
  \end{subfigure}
  \begin{center}
    \begin{subfigure}{0.47\textwidth}
      \centering
      \includegraphics[width=.5\textwidth]{precision_legend}
      \caption{Legend for the above figures}
      \label{fig:precision_legend}
    \end{subfigure}
  \end{center}
  \caption{Precision@5 and precision@10 with histogram and $\ell_2$ distance measures for the bag of words retrieval system.  Results are shown for three different images, and their average scores.  The legend for all graphs is provided in Figure~\ref{fig:precision_legend}.}
  \label{fig:precision}
\end{figure}

\begin{table}
  \begin{subtable}{0.47\textwidth}
    \begin{tabular}{l|l}
      \texttt{all\_souls\_000006} & $0.2$ \\
      \texttt{all\_souls\_000065} & $0.8$ \\
      \texttt{all\_souls\_000075} & $0.2$ \\
      \hline
      Average                     & $0.12$ \\
    \end{tabular}
    \caption{Precision@5, histogram distance.}
    \label{tab:color_p5h}
  \end{subtable}
  \hspace*{\fill}
  \begin{subtable}{0.47\textwidth}
    \begin{tabular}{l|l}
      \texttt{all\_souls\_000006} & $0.2$ \\
      \texttt{all\_souls\_000065} & $0.6$ \\
      \texttt{all\_souls\_000075} & $0.0$ \\
      \hline
      Average                     & $0.27$ \\
    \end{tabular}
    \caption{Precision@5, $\ell_2$ distance.}
    \label{tab:color_p5l}
  \end{subtable}
  \begin{subtable}{0.47\textwidth}
    \begin{tabular}{l|l}
      \texttt{all\_souls\_000006} & $0.1$ \\
      \texttt{all\_souls\_000065} & $0.4$ \\
      \texttt{all\_souls\_000075} & $0.0$ \\
      \hline
      Average                     & $0.17$ \\
    \end{tabular}
    \caption{Precision@10, histogram distance.}
    \label{tab:color_p10h}
  \end{subtable}
  \hspace*{\fill}
  \begin{subtable}{0.47\textwidth}
    \begin{tabular}{l|l}
      \texttt{all\_souls\_000006} & $0.1$ \\
      \texttt{all\_souls\_000065} & $0.5$ \\
      \texttt{all\_souls\_000075} & $0.2$ \\
      \hline
      Average                     & $0.27$ \\
    \end{tabular}
    \caption{Precision@10, $\ell_2$ distance.}
    \label{tab:color_p10l}
  \end{subtable}
  \caption{Precision@5 and precision@10 with histogram and $\ell_2$ distance measures for the color histogram retrieval system.  Results are shown for three different images, and their average scores.}
  \label{tab:color_precision}
\end{table}

\end{document}
