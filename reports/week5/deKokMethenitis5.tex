\documentclass[a4paper,10pt,twoside]{article}

\usepackage[english]{babel}
\usepackage{amssymb}
\usepackage{fancyhdr}
\usepackage{graphicx}
\usepackage[ocgcolorlinks]{hyperref}
\usepackage{todonotes}
\usepackage{caption}
\usepackage{subcaption}

\pagestyle{fancy}
\fancyhead{}
\fancyfoot{}
\fancyhead[RE,LO]{de Kok \& Methenitis}
\fancyhead[RO,LE]{Week 5 \& 6}
\fancyfoot[RE]{\thepage$\quad \square$}
\fancyfoot[LO]{$\square \quad$\thepage}

\title{Report for Week 5 \& 6\\\normalsize Create your own ``Google Similar Images'' system}

\author{Patrick de Kok (5640318) \and Georgios Methenitis (10407537)}

\begin{document}
\maketitle
\thispagestyle{empty}

\section{Implement $k$-NN classifier}

\section{Different $k$-NN strategy}
One possibility would be to return a vector of probabilities, where each element represents the probability that the query image belongs to that class.  The probability of a label will be computed as the relative frequency of that label among the $k$ nearest neighbors in the training images.  Smoothing methods may be applied, such as Good-Turing smoothing, such that the probability of the query image having one of the labels that did not occur among the test images, is not zero.

This will give you a probability distribution $P(\mbox{label} | \mbox{query})$, which can be more meaningful for the end user of the system.  For instance, where binary classifiers will tell you that an image represents a label falsely, this classifier will present a low probability.  This will express the classifier's low certainty of the label belonging to the image.

A version of this, without Good-Turing smoothing has been implemented. % Bluff! Will do so.

\section{Implement mean class accuracy}

\section{Cross validation of $k$}

\section{Evaluation of 4 possible classifiers}

\section{Use scikit}

\section{Ring data -- linear SVM}

\section{Non-linear transformations}

\section{Transform to polar coordinates}

\section{Non-linear SVM}

\section{Image classification -- linear SVM}

\section{Image classification -- discussing the results}

\section{Image classification -- comparison $k$-NN vs SVM}


%\bibliographystyle{plain}
%\bibliography{thereference}
\end{document}
